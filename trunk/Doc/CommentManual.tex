\documentclass[11pt,a4paper]{article}

\usepackage{fontspec}
\usepackage{xunicode}
\usepackage{xltxtra}
\usepackage{xeCJK}
\usepackage[pdfstartview=FitH, CJKbookmarks=true, bookmarksnumbered=true, bookmarksopen=true]{hyperref}
\usepackage[top=1.2in,bottom=1.2in,left=1.2in,right=1in]{geometry}
\usepackage{titlesec}
\usepackage{listings}
\usepackage{xcolor} % 使用颜色宏包

\XeTeXlinebreaklocale "zh"
\XeTeXlinebreakskip = 0pt plus 1pt minus 0.1pt

\punctstyle{quanjiao}

\setmainfont{Times New Roman}	% 缺省英文字体
\setsansfont{Arial}				% 缺省英文非衬线字体
\setmonofont{DejaVu Sans Mono}	% 缺省英文等宽字体

\setCJKmainfont[BoldFont=黑体,ItalicFont=华文新魏]{华文中宋}	% 缺省中文字体
\setCJKsansfont[BoldFont=黑体]{华文细黑}		% 缺省中文非衬线字体
\setCJKmonofont{微软雅黑}									% 缺省中文等宽字体,中文字体均等宽。

\lstset{
basicstyle=\small\ttfamily,						%基础风格
tabsize=4,										%TAB键宽度
keywordstyle=\color{blue!50!black}\bfseries,	%关键词风格
commentstyle=\color{green!50!black},			%注释风格
frame=shadowbox,								%外框风格
rulesepcolor=\color{white!50!black},			%阴影颜色
xleftmargin=2em,								%左间距
xrightmargin=2em,								%右间距
escapeinside=``
}

\hypersetup{  
  pdfinfo={  
    Title={C语言注释规范},  
    Subject={编写符合doxygen标准的C语言注释},  
    Author={刘宝},  
  }  
}

\begin{document}
\title{\bf C语言注释规范\\[10pt]\normalsize 版本 1.0}
\author{刘宝}
\date{\today}
\maketitle
本规范旨在对使用C语言编写的Doxygen注释统一注释风格。
\section{准备工作}
\subsection{中文支持}
\begin{center}
\begin{tabular}{l @{ = } l}
\textbf{DOXYFILE\_ENCODING} & \textbf{GB2312}\\
\textbf{INPUT\_ENCODING} & \textbf{GB2312}\\
\textbf{OUTPUT\_LANGUAGE} & \textbf{Chinese}
\end{tabular}
\end{center}
\subsection{C语言支持}
\begin{center}
\begin{tabular}{l @{ = } l l}
\textbf{OPTIMIZE\_OUTPUT\_FOR\_C} & \textbf{TRUE} & \#对C语言优化\\
\textbf{TYPEDEF\_HIDES\_STRUCT} & \textbf{TRUE} & \# struct/union/enum 使用typedef的名字\\
\textbf{HIDE\_SCOPE\_NAMES} & \textbf{TRUE} & \#隐藏class和namespace作用域名\\
\textbf{TAB\_SIZE} & \textbf{4} & \#制表符占4个空格
\end{tabular}
\end{center}
\subsection{额外设置}
根据需要选择相应设置:\par
\begin{center}
\begin{tabular}{l @{ = } l l}
\textbf{SOURCE\_BROWSER} & \textbf{YES} & \#自动加入源码\\
\textbf{INLINE\_SOURCES} & \textbf{YES} & \#在函数文档中嵌入对应源码\\
\textbf{REFERENCES\_LINK\_SOURCE} & \textbf{NO} & \#把符号链接到文档\\
\textbf{REFERENCES\_LINK\_SOURCE} & \textbf{YES} & \#把符号链接到源码
\end{tabular}
\end{center}
\section{文件注释}
\subsection{样例}
\begin{lstlisting}[language={[ANSI]C}]
/*!
\file IMGProcess.h
\author LiuBao
\version 1.0
\date 2010/12/25
\brief The brief file comment
\details More details about this file, may be multi-line.
*/
\end{lstlisting}
\subsection{说明}
\begin{itemize}
\item “{$\backslash$file}”后的文件名需与当前文件名一致
\end{itemize}
\section{结构体注释}
\subsection{简洁样例}
\begin{lstlisting}[language={[ANSI]C}, morekeywords={uint8_t}]
typedef struct _PrimVolDesc			///  The brief struct comment
{
	uint8_t PrimaryIndicator;		///< The brief member comment
	uint8_t ISO9660Identifier[5];	///< The brief member comment
}PrimVolDesc;
\end{lstlisting}
\subsection{详细样例}
\begin{lstlisting}[language={[ANSI]C}, morekeywords={uint8_t}]
/*! More details about this struct, may be multi-line. */
typedef struct _PrimVolDesc			///  The brief struct comment
{
	uint8_t PrimaryIndicator;		///< The brief member comment
	
	/*! More details about this member, may be multi-line. */
	uint8_t ISO9660Identifier[5];	///< The brief member comment
}PrimVolDesc;
\end{lstlisting}
\subsection{说明}
\begin{itemize}
\item “{///}”与注释间留有2个空格
\item “{///<}”与注释间留有1个空格
\item 结构体成员的详细注释写在该成员上面
\item 结构体成员的详细注释与上一成员间留1个空行
\item 推荐结构体使用typedef类型定义
\item 推荐使用简洁的结构体注释
\end{itemize}
\section{枚举注释}
\subsection{样例}
\begin{lstlisting}[language={[ANSI]C}]
typedef enum _COLOR	///  The brief enum comment
{
	BLACK,			///< The brief member comment
	NAVY,			///< The brief member comment
}COLOR;
\end{lstlisting}
\subsection{说明}
\begin{itemize}
\item 与结构体的简洁注释相同
\end{itemize}
\section{宏注释}
\subsection{简洁样例}
\begin{lstlisting}[language={[ANSI]C}]
#define GarbageByte 0xCC	///< The brief macro comment
#define EndByte 0xE1		///< The brief macro comment

/// This macro is too long, so comment here briefly!
#define MALLOC(size) Mem_alloc(size, __FILE__, __LINE__)
\end{lstlisting}
\subsection{详细样例}
\begin{lstlisting}[language={[ANSI]C}]
/*!
The detail macro comment, may be multi-line.
\param a The brief parameter comment
\param b The brief parameter comment
\return The brief return value comment
*/
#define MAX(a, b) ((a) > (b)) ? (a) : (b)
\end{lstlisting}
\subsection{说明}
\begin{itemize}
\item 尽量少写宏函数,可以使用内联函数代替
\item 推荐使用简洁的宏注释
\end{itemize}
\section{函数注释}
\subsection{简洁样例}
\begin{lstlisting}[language={[ANSI]C}]
/*! The detail function comment, may be multi-line. */
void PrintAllColor();
\end{lstlisting}
\subsection{详细样例}
\begin{lstlisting}[language={[ANSI]C}, morekeywords={media_t}]
/*!
The detail function comment, may be multi-line.
\param media The brief parameter comment
\return The brief return value comment
*/
int CheckIMGFileSystem(media_t media);
\end{lstlisting}
\subsection{说明}
\begin{itemize}
\item 参数简述和返回值简述不写句号
\item 无参无返回值的简单函数使用简洁注释
\item 头文件有声明的函数注释在头文件中;否则注释在代码文件中
\end{itemize}
\section{其它注释}
代码中其余注释一律使用普通的单行注释“{//}”和多行注释“{/*}”“{*/}”。
\section{附录}
\subsection{注释换行}
Doxygen会忽略你注释中的换行符,将多行注释连接成一个连续行并使用空格隔开。如果你希望保留两行注释之间的换行,需要在该行末加入“{$\backslash$n}”。
\subsection{常用命令}
\begin{center}
\begin{tabular}{l c l}
\centering\textbf{命令} & \textbf{生成字段名} & \multicolumn{1}{c}{\textbf{说明}}\\
\hline
$\backslash$attention & 注意 &\\
$\backslash$author & 作者 &\\
$\backslash$bug & 缺陷 & 链接到所有缺陷汇总的缺陷列表\\
$\backslash$brief & & 简要注释\\
$\backslash$code & & 代码块开始,与“{$\backslash$endcode}”成对使用\\
$\backslash$details & & 详细注释\\
$\backslash$date & 日期 &\\
$\backslash$endcode & & 代码块结束,与“{$\backslash$code}”成对使用\\
$\backslash$file & <文件名>文件参考 & 用在文件注释中\\
$\backslash$param & 参数 & 用在函数注释中\\
$\backslash$return & 返回 & 用在函数注释中\\
$\backslash$todo & TODO & 链接到所有TODO汇总的TODO列表\\
$\backslash$version & 版本 &\\
$\backslash$warning & 警告 &
\end{tabular}
\end{center}
\end{document}